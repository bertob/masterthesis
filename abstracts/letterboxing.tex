\documentclass{template/sigchi}

% Use this command to override the default ACM copyright statement
% (e.g. for preprints).  Consult the conference website for the
% camera-ready copyright statement.

%% EXAMPLE BEGIN -- HOW TO OVERRIDE THE DEFAULT COPYRIGHT STRIP -- (July 22, 2013 - Paul Baumann)
\toappear{Research abstract. \\
Last change \today, \currenttime\\
tobias@bernard.im\\
david.lindlbauer@tu-berlin.de}
%% EXAMPLE END -- HOW TO OVERRIDE THE DEFAULT COPYRIGHT STRIP -- (July 22, 2013 - Paul Baumann)

% Arabic page numbers for submission.  Remove this line to eliminate
% page numbers for the camera ready copy
 \pagenumbering{arabic}

% Load basic packages
\usepackage{balance}  % to better equalize the last page
\usepackage{graphics} % for EPS, load graphicx instead 
\usepackage[T1]{fontenc}
\usepackage{txfonts}
\usepackage{mathptmx}
\usepackage[pdftex]{hyperref}
\usepackage{color}
\usepackage{booktabs}
\usepackage{textcomp}
\usepackage{datetime}
% Some optional stuff you might like/need.
\usepackage{microtype} % Improved Tracking and Kerning
% \usepackage[all]{hypcap}  % Fixes bug in hyperref caption linking
\usepackage{ccicons}  % Cite your images correctly!
% \usepackage[utf8]{inputenc} % for a UTF8 editor only
%\usepackage{todonotes}
%********CUSTOM COMMANDS**********%

% create a shortcut to typeset table headings
\newcommand\tabhead[1]{\small\textbf{#1}}

\definecolor{brown}{rgb}{0.59, 0.29, 0.0}

% COMMANDS FOR NOTES PER PERSON
\newcommand\david[1]{\textit{\textcolor{red}{#1}}}
\newcommand\marc[1]{\textit{\textcolor{green}{#1}}}
%\newcommand\remove[1]{\textit{\textcolor{yellow}{#1}}}

\newcommand\todo[1]{\textit{\textcolor{brown}{[todo]~#1}}}

%\newcommand\david[1]{}
%\newcommand\marc[1]{}
\newcommand\remove[1]{}

%\newcommand{\comment}[1]{}

\usepackage{tabularx}
\usepackage[percent]{overpic}

\usepackage{nameref}

\usepackage{xspace}
\usepackage{enumitem}
\usepackage{mathtools}
\usepackage{commath}
\DeclareMathOperator*{\argmin}{argmin}

\usepackage{amssymb}% http://ctan.org/pkg/amssymb
\usepackage{pifont}% http://ctan.org/pkg/pifont
\newcommand{\cmark}{\ding{51}}%
\newcommand{\xmark}{\ding{53}}%

\newcommand{\customtilde}{{\raise.17ex\hbox{$\scriptstyle\sim$}}}

\newcommand{\mv}[1]{\mathbf{#1}}
\def \R {\mathbb{R}}
\def \Z {\mathbb{Z}}
\def \tp {^{\mathsf{T}}}

\newcommand{\etal}{et~al.\xspace}
\newcommand{\eg}{e.\,g.\xspace}
\newcommand{\ie}{i.\,e.\xspace}
\newcommand{\cf}{cf.\xspace}
\newcommand{\etc}{etc.\xspace}

\newcommand{\subparagraph}[1]{\textit{{#1}}:~}

%********END CUSTOM COMMANDS**********%

\def\plaintitle{}
\def\plainauthor{}
\def\emptyauthor{}
\def\plainkeywords{}
\def\plaingeneralterms{}

% llt: Define a global style for URLs, rather that the default one
\makeatletter
\def\url@leostyle{%
  \@ifundefined{selectfont}{
    \def\UrlFont{\sf}
  }{
    \def\UrlFont{\small\bf\ttfamily}
  }}
\makeatother
\urlstyle{leo}

% To make various LaTeX processors do the right thing with page size.
\def\pprw{8.5in}
\def\pprh{11in}
\special{papersize=\pprw,\pprh}
\setlength{\paperwidth}{\pprw}
\setlength{\paperheight}{\pprh}
\setlength{\pdfpagewidth}{\pprw}
\setlength{\pdfpageheight}{\pprh}

\definecolor{linkColor}{RGB}{6,125,233}
\hypersetup{%
  pdftitle={\plaintitle},
% Use \plainauthor for final version.
%  pdfauthor={\plainauthor},
  pdfauthor={\emptyauthor},
  pdfkeywords={\plainkeywords},
  bookmarksnumbered,
  pdfstartview={FitH},
  colorlinks,
  citecolor=black,
  filecolor=black,
  linkcolor=black,
  urlcolor=linkColor,
  breaklinks=true,
}

\begin{document}

\title{Beyond letterboxing: Spatial interface patterns for navigating large data sets in VR}\maketitle

\section{Abstract}
When presenting large amounts of information in user interfaces, cases where the content exceeds the size of the viewport need to be accounted for. In screen-based interfaces, these cases are often handled by using pagination, scrolling or zooming. All of the above methods rely on letterboxing, \ie they employ a virtual space on which all of the information is laid out, and cut off the content at the edges of the screen. This technique works on screens because of their inherently fixed size and the resulting natural boundaries at the edges. It is not ideal, however, because content which is outside the viewport is completely invisible, and there is no spatial sense of position. In Virtual Reality there are no boundaries, interfaces can take up the entire field of vision. However, since this field of vision is not infinitely large, there are still limits to how much data can be viewed at any one time. It would not make sense to use letterboxing to hide the additional information, because there are no natural boundaries in VR.

We propose a new paradigm for showing large amounts of information in \textsc{vr} environments, which does not rely on letterboxing. We explore various spatial models and interaction techniques which enable the presentation and navigation of large data sets, while giving users a more explicit mental model of the data. For example, instead of disappearing outside the viewport, list items could stack up on top of a list, shrink to a tiny size, arrange themselves in a grid pattern, or a combination of all the above.

We will construct a series of Virtual Reality scenes to test the different approaches with the same data, in oder to compare them effectively. Our implementation will make use of a \textsc{vr} headset with positional tracking and two hand controllers to manipulate items in \textsc{vr}. However, due to the low resolution of commercially available \textsc{vr} headsets, we will also evaluate the viability of using a \textsc{cave}, which could potentially provide a higher resolution.

We will evaluate our solutions by running a qualitative user study, wherein users navigate our proposed spatial interfaces, as well as a letterboxing-based solution, in order to compare their usability. We will observe user behavior during the study and collect feedback using a questionnaire afterwards.

\subsection{Contributions}
\begin{itemize}
  \setlength\itemsep{.1em}
  \item We will develop and evaluate general-purpose patterns for displaying and navigating large amounts of information in \textsc{vr}, which can be used by others building information-dense \textsc{vr} applications in the future
	\item Some of our work may be applicable to 2D interfaces trying to avoid the negative effects of letterboxing as well
\end{itemize}


\section{Notes \& ideas}
\begin{itemize}
  \setlength\itemsep{.1em}
	\item A card stack metaphor is probably the most obvious approach to try
	\item Exponentially decreasing element size around the currently viewed element (a bit like on the Apple Watch homescreen) could help accomodate a lot more data
	\item Switching representations (e.g. from full content to just the title to just an icon) to make elements further from the currently viewed one smaller
\end{itemize}

\bibliographystyle{SIGCHI-Reference-Format}
\bibliography{references}

\end{document}
